\documentclass{beamer}

\usepackage[utf8]{inputenc}
\usepackage{amsmath}
\usepackage{amsfonts}
\usepackage{amssymb}
\usepackage{graphicx}
\usepackage{hyperref}
\usepackage{xcolor}
\usepackage{listings}

% Define Julia language for listings
\lstdefinelanguage{Julia}{
  morekeywords={
    function, end, if, else, elseif, while, for, in, return, break, continue,
    local, global, const, let, do, struct, mutable, import, using, where,
    println, begin, try, catch
  },
  sensitive=true,
  morecomment=[l]\#,
  morestring=[b]",
}

\lstset{
  language=Julia,
  basicstyle=\ttfamily\small,
  keywordstyle=\color{blue}\bfseries,
  commentstyle=\color{gray}\em,
  stringstyle=\color{red},
  showstringspaces=false,
  numbers=left,
  numberstyle=\tiny\color{gray},
  breaklines=true,
  frame=single,
  tabsize=2,
  columns=fullflexible
}

\title[Julia for Data Science]{Julia: Bridging the Gap in Technical Computing for Data Science \& Beyond}
\author[Nick Uhorchak]{Nick Uhorchak \ University of Arkansas}
\date{\today}

\begin{document}

\begin{frame}
    \titlepage
\end{frame}

\begin{frame}{Outline}
    \tableofcontents
\end{frame}

\section{Introduction}

\begin{frame}{The Language Problem}
    \begin{itemize}
        \item \textbf{A familiar pain point:} But C++ is faster than ... \textit{(R, Python...)}.
        \item When performance is critical, often re-implementing in lower-level languages (C++, Fortran).
        \item This creates a development overhead and potential for bugs.
        \item Besides, who likes to write code twice!
    \end{itemize}
\end{frame}

\begin{frame}{What is Julia?}
    \begin{itemize}
        \item High-level, high-performance \textbf{dynamic} programming language.
        \item Designed for numerical and scientific computing.
        \item \textbf{"Walks like Python, Runs like C."}
        \item Open-source and free.
    \end{itemize}
\end{frame}

\begin{frame}{Why Julia Now? Key Interests}
    \begin{itemize}
        \item \textbf{Performance:} JIT compilation to native machine code (LLVM). Often competitive with C/Fortran.
        \item \textbf{Solves the "Language Problem":} Write high-performance code directly in Julia.
        \item \textbf{Mathematical Syntax:} Intuitive for mathematicians and engineers. Similar to python.
        \item \textbf{Dynamic \& Flexible:} Interactivity of dynamic languages with performance of compiled ones.
        \item \textbf{Growing Ecosystem:} Rich set of packages for data science, ML, scientific computing.
        \item \textbf{General Purpose:} Beyond numerical computing, capable of web dev, scripting.
    \end{itemize}
\end{frame}

\begin{frame}{Julia vs Python/R: What Really Matters for Data Science}
\textbf{What do you want as a data scientist?}

\begin{itemize}
    \item Fast iteration: load, transform, visualize, model — quickly.
    \item Simple syntax and interactive experience (like Python or R).
    \item Seamless scaling: small prototype to big data or production.
\end{itemize}

\vspace{0.4cm}
\textbf{Where Julia Fits:}

\begin{columns}
    \begin{column}{0.48\textwidth}
    \textbf{Julia}
    \scriptsize
    \begin{itemize}\setlength{\itemsep}{2pt}
        \item Feels like Python or R in the REPL.
        \item You get real speed without changing your code. (unless you code poorly anyways!)
        \item Same package handles exploration \& production.
        \item DataFrames.jl, CSV.jl, MLJ.jl, Plots.jl — clean and expressive.
    \end{itemize}
    \end{column}

    \begin{column}{0.48\textwidth}
    \textbf{Python / R}
    \scriptsize
    \begin{itemize}\setlength{\itemsep}{2pt}
        \item Excellent for exploration and prototyping.
        \item Performance usually depends on compiled extensions (NumPy, data.table).
        \item Scaling up often means changing tools.
        \item Language boundaries can complicate deployment.
    \end{itemize}
    \end{column}
\end{columns}

\vspace{0.3cm}
\textit{Julia offers the simplicity of scripting with the performance of systems code — in a single, consistent toolchain.}
\end{frame}

\section{Julia Fundamentals: Data Structures}

\begin{frame}[fragile]{Numbers and Arrays: Julia vs Python}
\begin{itemize}
    \item \textbf{Numbers:} Native support for various types (Integers, Floats, Complex, Rationals).
    \item \textbf{Arrays/Matrices:} Core for numerical computing.
    \item \textbf{One-based indexing:} (Similar to R/MATLAB, different from Python/C++).
    \item \textbf{Broadcasting (`.`):} Element-wise operations (similar to NumPy).
\end{itemize}

\vspace{0.3cm}
\begin{columns}
    \begin{column}{0.5\textwidth}
    \textbf{Julia}
    \begin{lstlisting}
# Vector
v = [1, 2, 3, 4, 5]

# Matrix
M = [1 2; 3 4]

# Array comprehension
A = [i*j for i in 1:3, j in 1:3]
    \end{lstlisting}
    \end{column}

    \begin{column}{0.5\textwidth}
    \textbf{Python (NumPy)}
    \begin{lstlisting}[language=Python]
import numpy as np

v = np.array([1, 2, 3, 4, 5])
M = np.array([[1, 2], [3, 4]])
A = np.array([[i*j for j in range(1, 4)] 
              for i in range(1, 4)])
    \end{lstlisting}
    \end{column}
\end{columns}
\end{frame}

\begin{frame}[fragile]{Tuples and Dictionaries: Julia vs Python}
\begin{itemize}
    \item \textbf{Tuples:} Immutable, ordered collections.
    \item \textbf{Dictionaries (Dict):} Key-value pairs.
\end{itemize}

\vspace{0.3cm}
\begin{columns}
    \begin{column}{0.48\textwidth}
    \textbf{Julia}
    \begin{lstlisting}
# Tuple
t = (1, "hello", 3.14)

# Dictionary
d = Dict("name" => "Julia",
         "version" => 1.9)
    \end{lstlisting}
    \end{column}

    \begin{column}{0.48\textwidth}
    \textbf{Python}
    \begin{lstlisting}[language=Python]
# Tuple
t = (1, "hello", 3.14)

# Dictionary
d = {"name": "Julia",
     "version": 1.9}
    \end{lstlisting}
    \end{column}
\end{columns}
\end{frame}

\begin{frame}[fragile]{Tabular Data: Julia vs Python}
\begin{itemize}
    \item Equivalent to R’s \texttt{data.frame} or Python’s Pandas DataFrame.
    \item Efficient for manipulation, supports missing data.
\end{itemize}

\vspace{0.3cm}
\begin{columns}
    \begin{column}{0.48\textwidth}
    \textbf{Julia (DataFrames.jl)}
    \begin{lstlisting}
using DataFrames

df = DataFrame(Name=["Alice", "Bob"],
               Age=[25, 30])

select(df, :Name)
filter(:Age => >(26), df)
    \end{lstlisting}
    \end{column}

    \begin{column}{0.48\textwidth}
    \textbf{Python (pandas)}
    \begin{lstlisting}[language=Python]
import pandas as pd

df = pd.DataFrame({
    "Name": ["Alice", "Bob"],
    "Age": [25, 30]
})

df[["Name"]]
df[df["Age"] > 26]
    \end{lstlisting}
    \end{column}
\end{columns}
\end{frame}

%%%%%%%%%%%%%
\begin{frame}[fragile]{Control Flow Syntax: The Role of \texttt{end}}

\textbf{Julia requires explicit \texttt{end} statements} to close control blocks like \texttt{for}, \texttt{if}, and \texttt{function}.

\vspace{0.4cm}
\begin{columns}[t]
    \begin{column}{0.33\textwidth}
    \textbf{Julia}
\begin{lstlisting}[language=Julia]
for i in 1:5
    println(i)
end
\end{lstlisting}
    \end{column}

    \begin{column}{0.33\textwidth}
    \textbf{Python}
\begin{lstlisting}[language=Python]
for i in range(1, 6):
    print(i)
\end{lstlisting}
    \end{column}

    \begin{column}{0.33\textwidth}
    \textbf{R}
\begin{lstlisting}[language=R]
for (i in 1:5) {
  print(i)
}
\end{lstlisting}
    \end{column}
\end{columns}

\vspace{0.3cm}
\begin{itemize}
    \item Julia uses \texttt{end} to clearly mark block boundaries.
    \item Python relies on indentation.
    \item R uses braces \texttt{\{\}} to define scope.
\end{itemize}

\end{frame}


%%%%%%%%%%%%%%


\section{Intro to Stats/ML/Viz Libraries}

\begin{frame}[fragile]{Statistics Libraries}
\begin{itemize}
    \item \texttt{Statistics.jl}, \texttt{StatsBase.jl}, \texttt{Distributions.jl}, \texttt{HypothesisTests.jl}
\end{itemize}
\begin{lstlisting}
using Statistics, Distributions

data = randn(100)
println("Mean: $(mean(data))")
println("Std Dev: $(std(data))")

d = Normal(0, 1)
println("PDF at 0: $(pdf(d, 0.0))")
\end{lstlisting}
\end{frame}

\begin{frame}[fragile]{Machine Learning with MLJ.jl}
\begin{itemize}
    \item \textbf{MLJ.jl:} Unified interface to many ML models.
    \item Common API for EvoTrees, DecisionTree, Flux, etc.
\end{itemize}
\begin{lstlisting}
using MLJ, EvoTrees, DataFrames
using MLJBase
@load EvoTreeClassifier pkg=EvoTrees

X = DataFrame(f1 = rand(100), f2 = rand(100))
y = categorical(rand(Bool, 100))

train, test = partition(eachindex(y), 0.7, shuffle=true)
Xtrain, Xtest = X[train,:], X[test,:]
ytrain, ytest = y[train], y[test]

model = EvoTreesClassifier()
mach = machine(model, Xtrain, ytrain)
fit!(mach, verbosity=0)

yhat = predict(mach, Xtest)
\end{lstlisting}
\end{frame}

\begin{frame}[fragile]{Visualization Libraries}
\begin{itemize}
    \item \textbf{Plots.jl}, \textbf{Makie.jl}, \textbf{StatsPlots.jl}
\end{itemize}
\begin{lstlisting}
using Plots, StatsPlots

x = 1:10; y = rand(10)
plot(x, y, seriestype = :scatter,
     title = "My Scatter Plot",
     xlabel = "X-axis", ylabel = "Y-axis")

histogram(randn(1000), bins=50, title="Normal Distribution Histogram")
\end{lstlisting}
\end{frame}

\section{Julia-Specific IDEs \& Best Practices}

\begin{frame}{IDEs for Julia Development}
\begin{itemize}

    \item \textbf{VS Code with Julia Extension}
    \item \textbf{Jupyter Notebooks via IJulia.jl}
    \item \textbf{Julia REPL}
    \item \textit{Positron (my favorite)}
\end{itemize}
    \begin{center}
        \includegraphics[width=0.8\textwidth]{positron.png}
    \end{center}
\end{frame}

\begin{frame}{Package Management: Julia vs Python vs R}
\textbf{How do I install and manage packages?}

\vspace{0.3cm}
\begin{table}[h]
\centering
\resizebox{\textwidth}{!}{%
\begin{tabular}{|l|c|c|c|}
\hline
\textbf{Feature} & \textbf{Julia} & \textbf{Python} & \textbf{R} \\
\hline
Built-in environment management & \texttt{Pkg} & \texttt{venv}, \texttt{virtualenv} & \texttt{renv}, \texttt{packrat} \\
\hline
Third-party tools & -- & \texttt{conda}, \texttt{pipenv}, \texttt{poetry} & \texttt{conda}, \texttt{checkpoint} \\
\hline
Dependency file & \texttt{Project.toml} & \texttt{requirements.txt}, \texttt{pyproject.toml} & \texttt{renv.lock} \\
\hline
Version pinning & \texttt{Manifest.toml} & \texttt{pip freeze}, \texttt{lock files} & \texttt{renv.lock} \\
\hline
Project isolation & \checkmark & \checkmark & \checkmark \\
\hline
Ease of use & High & Medium--High & Medium \\
\hline
\end{tabular}%
}
\caption{Comparison of environment and package management across Julia, Python, and R}
\end{table}

\end{frame}



\begin{frame}{Best Practices: Writing Performant Julia Code}
\begin{itemize}
    \item Prefer functions over global scope.
    \item Ensure type-stable functions.
    \item Use \texttt{BenchmarkTools.jl} to profile performance.
    \item Embrace multiple dispatch.
\end{itemize}
\end{frame}

\section{Use Cases \& Why Julia Excels}

\begin{frame}{Use Cases: Statistical \& ML Analyses}
\begin{itemize}
    \item MCMC, simulation, ML pipelines.
    \item No need to rewrite in C++ for speed.
    \item Optimization
\end{itemize}
\end{frame}

\begin{frame}{Use Cases: Plotting \& Data Manipulation}
\begin{itemize}
    \item High-fidelity plots with Makie.
    \item Big-data manipulation with DataFrames.jl
\end{itemize}
\end{frame}

\begin{frame}{Use Cases: Delivering Analytical Products}
\begin{itemize}
    \item Web apps with Genie.jl, dashboards, REST APIs.
    \item Compile to native executables.
\end{itemize}
\end{frame}

\begin{frame}{Why Julia?}
\begin{itemize}
    \item \textbf{Data Scientists:} Faster iteration, production-ready.
    \item \textbf{Engineers:} Simulation, modeling.
    \item \textbf{Mathematicians:}  \textit{optimization}, linear algebra.
    \item \textbf{Computer Scientists:} Metaprogramming, compiler tools.
\end{itemize}
\end{frame}

\section{Live Code Demonstration}

\begin{frame}{Live Code Demonstration}
\textbf{Live demonstration will show Julia in comparison to pyhton and R.}
\end{frame}

\section{Q\&A and Resources}

\begin{frame}{Questions \& Discussion}
    \centering
    \Huge{Q\&A}
\end{frame}

\begin{frame}{Resources}
\begin{itemize}
    \item \href{https://julialang.org}{julialang.org}
    \item \href{https://discourse.julialang.org}{Julia Discourse}
    \item \href{https://juliahub.com/packages}{JuliaHub}
    \item JuliaAcademy, RCall.jl, PyCall.jl
\end{itemize}
\end{frame}

\begin{frame}{Thank You!}
    \centering
    \Huge{Thank You!}
    \vspace{1cm}
    \Large{Questions? Comments?}
    \vspace{0.5cm}
    \normalsize
    \href{mailto:nicholas.m.uhorchak.mil@army.mil}{nicholas.m.uhorchak.mil@army.mil}
\end{frame}

\end{document}
